%
% Sablona pro psani posudku bakalarskych praci (oboru Informatika na MFF UK)
% ==========================================================================
%
% Tento soubor je siren pod licenci Apache, verze 2.0 (viz soubor LICENSE).
% This file is distributed under Apache License, ver. 2.0 (see file LICENSE).
%
% Aktualni verzi tohoto souboru najdete na teto webove strance:
% https://github.com/D-iii-S/admin-forms
% 
% Tento soubor je pripraven v kodovani UTF-8. V pripade, ze vas system
% toto kodovani nepodporuje, muzete jej zmenit v ramci nacitani balicku
% inputenc nize. Nezapomente ale v takovem pripade zkontrolovat, ze se
% v poradku vysazi vsechny polozky s hacky a carky (komentare jsou z
% duvodu archaicke kompatibility zatim udrzovany bez hacku a carek).
%
% Tato sablona prepoklada, ze autor posudku zna rozumne system LaTeX a
% proto vyplnovani posudku probiha formou rozsirovani vlastniho tela
% dokumentu, nikoliv definici maker pro jednotlive casti jako u jinych
% podobnych sablon. Jednotlive casti jsou komentovany primo v TeXovem
% kodu, sablona jako takova je pak schopna graficky zvyraznit chybejici
% udaje i ve vyslednem PDF souboru.
%
% Pred finalnim odevzdanim posudku muze byt vhodne rucne upravit
% zalamovani stranek. Interni pouziti prostredi longtable zarucuje
% rozumne chovani ve vetsine pripadu; preciznejsi kontrolu nad mistem
% zalomeni (typicky pred zacatkem jednotlivych bloku) lze vynutit
% napr. vlozenim prikazu \pagebreak.
%
% Pokud nechcete menit vzhled dokumentu, preskocte nyni na zacatek tela
% dokumentu, tj. ke znacce \begin{document}.
%
\documentclass[12pt,a4paper]{article}

\usepackage[utf8x]{inputenc}


\usepackage[top=2cm,left=2cm,right=2cm,bottom=2.5cm]{geometry}

\usepackage{xcolor}
\usepackage{ifthen}
\usepackage{longtable}
\usepackage{setspace}
\usepackage{array}
\def\arraystretch{1.1}%

\newlength{\myparindent}
\setlength{\myparindent}{1em}

% Vlastni typy, pro P definujeme odsazeni odstavce.
\newcolumntype{C}[1]{>{\centering\let\newline\\\arraybackslash\hspace{0pt}}m{#1}}
\newcolumntype{R}[1]{>{\raggedleft\let\newline\\\arraybackslash\hspace{0pt}}m{#1}}
\newcolumntype{P}[1]{>{%
\let\newline\\\arraybackslash\hspace{\myparindent}%
\setlength{\parindent}{\myparindent}%
\renewcommand\noindent{\hspace{-\myparindent}}}p{#1}
}

% Zobrazi puvodni misto pouziti makra (radek) ve zdrojovem dokumentu.
\newcommand\showPosition{{\textcolor{red}{\footnotesize(řádek~\the\inputlineno)}}}

% Zobrazi dany text cervene.
\newcommand\errorText[1]{{\color{red} \textbf{#1}}}

% Vypise krizek, pokud znamka (parametr 2) je stejna jako parametr #1.
\newcommand\makeMark[2]{\ifthenelse{\equal{#1}{#2}}{X}{~}}

% Polozka v hlavicce.
% \field{Jmeno pole}{Hodnota}
% Pokud je hodnota rovna X, je pole vysviceno jako chybne.
\newcommand{\field}[2]{
\ifthenelse{\equal{#2}{X}}{\errorText{#1} & \showPosition}{%
    \ifthenelse{\equal{#2}{X \hfill Oponent}}{\errorText{#1} & \showPosition \hfill Oponent}{%
        \ifthenelse{\equal{#2}{X \hfill Vedoucí}}{\errorText{#1} & \showPosition \hfill Vedoucí}{%
        \textbf{#1}~~ & #2%
        }%
    }%
}%
 \\
}

% Hlavicka prace.
% Predpokladame, ze pouze polozky \field budou uvnitr tohoto prostredi.
\newenvironment{topmatter}{%
\begin{center}
{\Large Posudek bakalářské práce}

\vspace{3mm}

{\large Matematicko-fyzikální fakulta Univerzity Karlovy v Praze}
\end{center}

\renewcommand\medskip{\\}
\begin{tabular}{R{4cm}p{9cm}}
}{%
\end{tabular}

\vspace{2em}
}


% Vlozi dilci hodnoceni.
% \grade{Nazev}{Znamka jako cislo 1-4}{Detailnejsi popis}
\newcommand{\grade}[3]{
\hspace{-\parindent}%
\hspace{-3mm}
\begin{tabular}{p{11.1cm}|C{.8cm}|C{.8cm}|C{.8cm}|C{1.8cm}}
    \ifthenelse{\equal{#2}{X}}{\color{red} \textbf{#1}}{#1}
        \hfill
        \textit{\scriptsize \ifthenelse{\equal{#3}{}}{}{~\ldots$\,$#3}}
    & \makeMark{1}{#2}
    & \makeMark{2}{#2}
    & \makeMark{3}{#2}
    & \ifthenelse{\equal{#2}{X}}{\showPosition}{\makeMark{4}{#2}}
    \\
\end{tabular}
\\
\hline%
}


% Blok s hodnocenim.
\newenvironment{evaluation}[1]{%
\noindent
\begin{tabular}{p{11.1cm}C{.8cm}C{.8cm}C{.8cm}C{1.8cm}}
\hspace{-2ex} \textbf{#1} & \footnotesize lepší & \footnotesize OK & \footnotesize horší & \footnotesize nevyhovuje
\end{tabular}
\vspace{-1.5em}
\begin{longtable}{|P{17.2cm}|}
\hline%
}{%
\\
\hline
\end{longtable}
}


% Zaverecna znamka.
% \finalgrade{Znamka slovne}{Ano/Ne - doporuceni k zvlastnimu oceneni}
\newcommand{\finalgrade}[2]{
\begin{tabular}{R{8cm}p{4cm}}
\field{Celkové hodnocení}{#1}
\field{Práci navrhuji na zvláštní ocenění}{#2}
\end{tabular}
}


%%%%%%%%%%%%%%%%%%%%%%%%%%%%%%%%%%%%%%%%%%%%%%%%%%%%%%%%%%%%%%%%%
%                                                               %
%   ZDE zacnete vyplnovat vlastni hodnoceni bakalarske prace.   %
%                                                               %
%%%%%%%%%%%%%%%%%%%%%%%%%%%%%%%%%%%%%%%%%%%%%%%%%%%%%%%%%%%%%%%%%
%
% Prosím vyplňte hodnocení křížkem u každého kritéria.
% Hodnocení OK označuje práci, která kritérium vhodným způsobem splňuje.
% Hodnocení lepší a horší % označují splnění nad a pod rámec obvyklý pro bakalářskou práci.
% Hodnocení nevyhovuje označuje práci, která by neměla být obhájena.
% Hodnocení v případě potřeby doplňte komentářem.
%
% Komentář prosím doplňte všude, kde je hodnocení jiné než OK.
%
\begin{document}

% Pouze pokud bude mit vase hodnoceni vice jak 2 stranky, ma smysl zapinat cislovani stranek.
\pagestyle{empty}


%
% Sazba hlavicky posudku
% ----------------------
%
% Vyplnte prislusne udaje misto hodnot X v druhem argumentu makra \field.
%
\begin{topmatter}

% Informace o autorovi a jeho studiu
\field{Autor práce}{X}
\field{Název práce}{X}
\field{Rok odevzdání}{X}
\field{Studijní program}{Informatika}
\field{Studijní obor}{X}

\medskip


\field{Autor posudku}{X \hfill Vedoucí}
%\field{Autor posudku}{X \hfill Oponent}
\field{Pracoviště}{X}
%\field{Pracoviště}{Informatický ústav Univerzity Karlovy}
%\field{Pracoviště}{Katedra aplikované matematiky}
%\field{Pracoviště}{Katedra distribuovaných a spolehlivých systémů}
%\field{Pracoviště}{Katedra softwarového inženýrství}
%\field{Pracoviště}{Katedra softwaru a výuky informatiky}
%\field{Pracoviště}{Katedra teoretické informatiky a matematické logiky}
%\field{Pracoviště}{Středisko informatické sítě a laboratoří}
%\field{Pracoviště}{Ústav formální a aplikované lingvistiky}

\end{topmatter}


%
% Souhrnne hodnocení cele prace
% -----------------------------
%
% Jednotliva hodnocení vyplnte jako cisla 1-4 v druhem argumentu makra \grade (misto znaku X).
% 1 oznacuje splneni kriteria nad urovni beznou pro bakalarske prace,
% 2 splneni na obvykle urovni,
% 3 splneni pod beznou urovni,
% 4 oznacuje zavazny nedostatek.
%
\begin{evaluation}{K celé práci}
\grade{Obtížnost zadání}{X}{}
\grade{Splnění zadání}{X}{}
\grade{Rozsah práce}{X}{textová i implementační část, zohlednění náročnosti}

% Do tohoto mista (pred konec prostredi evaluation) vlozte slovni komentar.
\end{evaluation}


%
% Hodnoceni textove casti prace
% -----------------------------
%
% Instrukce viz vyse.
%
\begin{evaluation}{Textová část práce}
\grade{Formální úprava}{X}{jazyková úroveň, typografická úroveň, citace}
\grade{Struktura textu}{X}{kontext, cíle, analýza, návrh, vyhodnocení, úroveň detailu}
\grade{Analýza}{X}{}
\grade{Vývojová dokumentace}{X}{}
\grade{Uživatelská dokumentace}{X}{}

% Slovni komentar.
\end{evaluation}


%
% Hodnoceni implementacni casti prace
% -----------------------------------
%
% Instrukce viz vyse.
%
\begin{evaluation}{Implementační část práce}
\grade{Kvalita návrhu}{X}{architektura, struktury a algoritmy, použité technologie}
\grade{Kvalita zpracování}{X}{jmenné konvence, formátování, komentáře, testování}
\grade{Stabilita implementace}{X}{}

% Slovni komentar.
\end{evaluation}


%
% Zaverecne hodnoceni - znamka
% ----------------------------
%
% Prvni radek (se znamkou X) slouzi pouze pro upozorneni ve vystupnim
% souboru. Po vybrani znamky jej smazte nebo zakomentujte. Ke znamce
% je mozne pripsat formulace jako "spíše lepší" či "spíše horší".
%
% Druhy parametr makra (Ano/Ne) se tyka navrzeni teto prace na zvlastni oceneni.
%
\finalgrade{X}{X}
%\finalgrade{Výborně}{Ano}
%\finalgrade{Výborně}{Ne}
%\finalgrade{Velmi dobře}{Ne}
%\finalgrade{Dobře}{Ne}
%\finalgrade{Neprospěl(a)}{Ne}


%
% Datum odevzdani posudku
% -----------------------
%
\vspace{2em}
Datum \hfill Podpis \hspace{5cm}

\end{document}
